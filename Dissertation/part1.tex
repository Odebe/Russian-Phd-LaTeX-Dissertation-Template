\chapter{Описание прикладной области}\label{ch:ch1}

\section{Интернет-хостинг}\label{sec:ch1/sec1}
Интернет-хостинг - это служба, которая предоставляет интернет-серверы, позволяя организациям и частным лицам размещать контент в Интернет. Существуют различные уровни обслуживания и различные виды предлагаемых услуг.

Распространенным видом хостинга является веб-хостинг. Большинство хостинг-провайдеров предлагают комбинацию услуг, например, хостинг электронной почты.

Чаще всего интернет-хостинг предоставляет сервер с хорошей пропуской способностью интернет соединения, на котором клиенты могут запускать любые приложения.

\subsection{Dedicated Server. Выделенный сервер.}\label{sec:ds_hosting}
Выделенный хостинг - это вариант хостинга в Интернете, в котором физический сервер (или серверы) выделяются для одного бизнес-клиента. Заказчик имеет полный контроль над машиной, поэтому он может оптимизировать ее для своих уникальных требований, включая производительность и безопасность. Хостинг-провайдер предоставляет физический сервер и среду, сопутствующие услуги и техническую поддержку.

\subsection{VPS. Виртуальный выделенный сервер}\label{sec:vps_hosting}
VPS - это сокращение от Virtual Private Server (Виртуальный выделенный сервер). VPS хостинг - одна из самых популярных услуг хостинга, которую можно выбрать для размещения сайта и приложений. Для работы используется технология виртуализации, позволяющая предоставить выделенные (частные) ресурсы на сервере с несколькими пользователями.

Это более безопасное и стабильное решение, чем виртуальный хостинг, где клиент не получает выделенного серверного пространства. Тем не менее, VPS меньше и дешевле, чем аренда всего сервера.

Хостинг VPS обычно выбирают владельцы веб-сайтов, у которых трафик среднего уровня превышает лимиты планов общего хостинга, но при этом ему не нужны ресурсы выделенного сервера, или энтузиасты, которые ипользуются выделенные сервера для своих приложений.

\subsection{Колокация}\label{sec:colo_hosting}
Колокация - услуга, состоящая в том, что клиенту предоставляется оборудование на своей территории (датацентре). В случаях, если оборудование тоже арендуется клиентом, это называется "арендой выделенного сервера".

Такое размещение позволяет сэкономить на организации связи. Чаще всего колокацию используются для веб-сайтов и другим сетевым службам с большим объемом трафика, а также оборудование, к которму необходимое  доступ из многих точкек, например, VPN-концентраторы, шлюзы IP-телефонии.

\subsection{Облачный хостинг}\label{sec:cloud_hosting}
Облачный хостинг - это тип веб-хостинга, который использует несколько разных серверов для балансировки нагрузки и максимизации времени безотказной работы. Вместо использования одного сервера веб-сайт может подключиться к «кластеру», который использует ресурсы из централизованного пула. Это означает, что даже если один сервер выходит из строя, другой начинает работать, что повышает отказоуствойчивость системы.

Визуализируйте облако можно представить сеть взаимосвязанных компьютеров. Чем больше компьютеров подключено в сети, тем больше ресурсов добавляется в общее облако.

С облачным хостингом клиент получает часть так называемого облачного кластера. В отличие от традиционного веб-хостинга, где клиент получает определенное количество места с одного сервера.

Основные преимущества облачного хостинга заключаются во времени безотказной работы, изолированности ресурсов, простоте масштабирования и выделенных внешних адресах.

\section{Виртуализация}\label{sec:ch1/sec2}
\subsection{Описание}\label{sec:virt_decs}
Виртуализация использует программное обеспечение для создания уровня абстракции над компьютерным оборудованием, которое позволяет разделить аппаратные элементы одного компьютера - процессоры, память, хранилище и т.д. - на несколько виртуальных компьютеров, обычно называемых виртуальными машинами (ВМ). Каждая виртуальная машина работает со своей собственной операционной системой (ОС) и ведет себя как независимый компьютер, даже если она работает только на части фактического компьютерного оборудования.

Из этого следует, что виртуализация позволяет более эффективно использовать физическое компьютерное оборудование и обеспечивает большую отдачу от вложений в оборудование организации.

Сегодня виртуализация является стандартной практикой в корпоративной ИТ-архитектуре. Это также технология, которая управляет экономикой облачных вычислений. Виртуализация позволяет пользователям облачных вычислений приобретать только те вычислительные ресурсы, которые им необходимы, когда им они необходимо, и экономически эффективно масштабировать эти ресурсы по мере роста их рабочих нагрузок.

\subsection{Преимущества}\label{sec:virt_pros}
\subsection{Компоненты}\label{sec:virt_components}
\begin{enumerate}
  \item Виртуальная машина
  \item Гипервизор
\end{enumerate}

\subsection{Виртуальная машина}\label{sec:vm}
\subsection{Гипервизор}\label{sec:hyp}

\subsection{Типы}\label{sec:virt_types}

\subsection{Безопасность}\label{sec:virt_types}


% \chapter{Оформление различных элементов}\label{ch:ch1}

% \section{Форматирование текста}\label{sec:ch1/sec1}

% Мы можем сделать \textbf{жирный текст} и \textit{курсив}.

% \section{Ссылки}\label{sec:ch1/sec2}

% \subsection{Форматирование чисел и размерностей величин}\label{sec:units}

% Числа форматируются при помощи команды \verb|\num|:
% \num{5,3};
% \num{2,3e8};
% \num{12345,67890};
% \num{2,6 d4};
% \num{1+-2i};
% \num{.3e45};
% \num[exponent-base=2]{5 e64};
% \num[exponent-base=2,exponent-to-prefix]{5 e64};
% \num{1.654 x 2.34 x 3.430}
% \num{1 2 x 3 / 4}.
% Для написания последовательности чисел можно использовать команды \verb|\numlist| и \verb|\numrange|:
% \numlist{10;30;50;70}; \numrange{10}{30}.
% Значения углов можно форматировать при помощи команды \verb|\ang|:
% \ang{2.67};
% \ang{30,3};
% \ang{-1;;};
% \ang{;-2;};
% \ang{;;-3};
% \ang{300;10;1}.

% Обратите внимание, что ГОСТ запрещает использование знака <<->> для обозначения отрицательных чисел
% за исключением формул, таблиц и~рисунков.
% Вместо него следует использовать слово <<минус>>.

% Размерности можно записывать при помощи команд \verb|\si| и \verb|\SI|:
% \si{\farad\squared\lumen\candela};
% \si{\joule\per\mole\per\kelvin};
% \si[per-mode = symbol-or-fraction]{\joule\per\mole\per\kelvin};
% \si{\metre\per\second\squared};
% \SI{0.10(5)}{\neper};
% \SI{1.2-3i e5}{\joule\per\mole\per\kelvin};
% \SIlist{1;2;3;4}{\tesla};
% \SIrange{50}{100}{\volt}.
% Список единиц измерений приведён в таблицах~\refs{tab:unit:base,
% tab:unit:derived,tab:unit:accepted,tab:unit:physical,tab:unit:other}.
% Приставки единиц приведены в~таблице~\ref{tab:unit:prefix}.

% С дополнительными опциями форматирования можно ознакомиться в~описании пакета \texttt{siunitx};
% изменить или добавить единицы измерений можно в~файле \texttt{siunitx.cfg}.

% \begin{table}
%     \centering
%     \captionstyle{\centering}
%     \caption{Основные величины СИ}\label{tab:unit:base}
%     \begin{tabular}{llc}
%         \toprule
%         Название  & Команда                & Символ         \\
%         \midrule
%         Ампер     & \verb|\ampere| & \si{\ampere}   \\
%         Кандела   & \verb|\candela| & \si{\candela}  \\
%         Кельвин   & \verb|\kelvin| & \si{\kelvin}   \\
%         Килограмм & \verb|\kilogram| & \si{\kilogram} \\
%         Метр      & \verb|\metre| & \si{\metre}    \\
%         Моль      & \verb|\mole| & \si{\mole}     \\
%         Секунда   & \verb|\second| & \si{\second}   \\
%         \bottomrule
%     \end{tabular}
% \end{table}

% \begin{table}
%   \small
%   \centering
%   \begin{threeparttable}% выравнивание подписи по границам таблицы
%     \caption{Производные единицы СИ}\label{tab:unit:derived}
%     \begin{tabular}{llc|llc}
%         \toprule
%         Название       & Команда                 & Символ              & Название & Команда & Символ \\
%         \midrule
%         Беккерель      & \verb|\becquerel|  & \si{\becquerel}     &
%         Ньютон         & \verb|\newton|  & \si{\newton}                                      \\
%         Градус Цельсия & \verb|\degreeCelsius| & \si{\degreeCelsius} &
%         Ом             & \verb|\ohm| & \si{\ohm}                                         \\
%         Кулон          & \verb|\coulomb| & \si{\coulomb}       &
%         Паскаль        & \verb|\pascal| & \si{\pascal}                                      \\
%         Фарад          & \verb|\farad| & \si{\farad}         &
%         Радиан         & \verb|\radian| & \si{\radian}                                      \\
%         Грей           & \verb|\gray| & \si{\gray}          &
%         Сименс         & \verb|\siemens| & \si{\siemens}                                     \\
%         Герц           & \verb|\hertz| & \si{\hertz}         &
%         Зиверт         & \verb|\sievert| & \si{\sievert}                                     \\
%         Генри          & \verb|\henry| & \si{\henry}         &
%         Стерадиан      & \verb|\steradian| & \si{\steradian}                                   \\
%         Джоуль         & \verb|\joule| & \si{\joule}         &
%         Тесла          & \verb|\tesla| & \si{\tesla}                                       \\
%         Катал          & \verb|\katal| & \si{\katal}         &
%         Вольт          & \verb|\volt| & \si{\volt}                                        \\
%         Люмен          & \verb|\lumen| & \si{\lumen}         &
%         Ватт           & \verb|\watt| & \si{\watt}                                        \\
%         Люкс           & \verb|\lux| & \si{\lux}           &
%         Вебер          & \verb|\weber| & \si{\weber}                                       \\
%         \bottomrule
%     \end{tabular}
%   \end{threeparttable}
% \end{table}

% \begin{table}
%   \centering
%   \begin{threeparttable}% выравнивание подписи по границам таблицы
%     \caption{Внесистемные единицы}\label{tab:unit:accepted}

%     \begin{tabular}{llc}
%         \toprule
%         Название        & Команда                 & Символ          \\
%         \midrule
%         День            & \verb|\day| & \si{\day}       \\
%         Градус          & \verb|\degree| & \si{\degree}    \\
%         Гектар          & \verb|\hectare| & \si{\hectare}   \\
%         Час             & \verb|\hour| & \si{\hour}      \\
%         Литр            & \verb|\litre| & \si{\litre}     \\
%         Угловая минута  & \verb|\arcminute| & \si{\arcminute} \\
%         Угловая секунда & \verb|\arcsecond| & \si{\arcsecond} \\ %
%         Минута          & \verb|\minute| & \si{\minute}    \\
%         Тонна           & \verb|\tonne| & \si{\tonne}     \\
%         \bottomrule
%     \end{tabular}
%   \end{threeparttable}
% \end{table}

% \begin{table}
%     \centering
%     \captionstyle{\centering}
%     \caption{Внесистемные единицы, получаемые из эксперимента}\label{tab:unit:physical}
%     \begin{tabular}{llc}
%         \toprule
%         Название                & Команда                 & Символ                 \\
%         \midrule
%         Астрономическая единица & \verb|\astronomicalunit| & \si{\astronomicalunit} \\
%         Атомная единица массы   & \verb|\atomicmassunit| & \si{\atomicmassunit}   \\
%         Боровский радиус        & \verb|\bohr| & \si{\bohr}             \\
%         Скорость света          & \verb|\clight| & \si{\clight}           \\
%         Дальтон                 & \verb|\dalton| & \si{\dalton}           \\
%         Масса электрона         & \verb|\electronmass| & \si{\electronmass}     \\
%         Электрон Вольт          & \verb|\electronvolt| & \si{\electronvolt}     \\
%         Элементарный заряд      & \verb|\elementarycharge| & \si{\elementarycharge} \\
%         Энергия Хартри          & \verb|\hartree| & \si{\hartree}          \\
%         Постоянная Планка       & \verb|\planckbar| & \si{\planckbar}        \\
%         \bottomrule
%     \end{tabular}
% \end{table}

% \begin{table}
%   \centering
%   \begin{threeparttable}% выравнивание подписи по границам таблицы
%     \caption{Другие внесистемные единицы}\label{tab:unit:other}
%     \begin{tabular}{llc}
%         \toprule
%         Название                  & Команда                 & Символ             \\
%         \midrule
%         Ангстрем                  & \verb|\angstrom| & \si{\angstrom}     \\
%         Бар                       & \verb|\bar| & \si{\bar}          \\
%         Барн                      & \verb|\barn| & \si{\barn}         \\
%         Бел                       & \verb|\bel| & \si{\bel}          \\
%         Децибел                   & \verb|\decibel| & \si{\decibel}      \\
%         Узел                      & \verb|\knot| & \si{\knot}         \\
%         Миллиметр ртутного столба & \verb|\mmHg| & \si{\mmHg}         \\
%         Морская миля              & \verb|\nauticalmile| & \si{\nauticalmile} \\
%         Непер                     & \verb|\neper| & \si{\neper}        \\
%         \bottomrule
%     \end{tabular}
%   \end{threeparttable}
% \end{table}

% \begin{table}
%   \small
%   \centering
%   \begin{threeparttable}% выравнивание подписи по границам таблицы
%     \caption{Приставки СИ}\label{tab:unit:prefix}
%     \begin{tabular}{llcc|llcc}
%         \toprule
%         Приставка & Команда                 & Символ      & Степень &
%         Приставка & Команда                 & Символ      & Степень   \\
%         \midrule
%         Иокто     & \verb|\yocto| & \si{\yocto} & -24     &
%         Дека      & \verb|\deca| & \si{\deca}  & 1         \\
%         Зепто     & \verb|\zepto| & \si{\zepto} & -21     &
%         Гекто     & \verb|\hecto| & \si{\hecto} & 2         \\
%         Атто      & \verb|\atto| & \si{\atto}  & -18     &
%         Кило      & \verb|\kilo| & \si{\kilo}  & 3         \\
%         Фемто     & \verb|\femto| & \si{\femto} & -15     &
%         Мега      & \verb|\mega| & \si{\mega}  & 6         \\
%         Пико      & \verb|\pico| & \si{\pico}  & -12     &
%         Гига      & \verb|\giga| & \si{\giga}  & 9         \\
%         Нано      & \verb|\nano| & \si{\nano}  & -9      &
%         Терра     & \verb|\tera| & \si{\tera}  & 12        \\
%         Микро     & \verb|\micro| & \si{\micro} & -6      &
%         Пета      & \verb|\peta| & \si{\peta}  & 15        \\
%         Милли     & \verb|\milli| & \si{\milli} & -3      &
%         Екса      & \verb|\exa| & \si{\exa}   & 18        \\
%         Санти     & \verb|\centi| & \si{\centi} & -2      &
%         Зетта     & \verb|\zetta| & \si{\zetta} & 21        \\
%         Деци      & \verb|\deci| & \si{\deci}  & -1      &
%         Иотта     & \verb|\yotta| & \si{\yotta} & 24        \\
%         \bottomrule
%     \end{tabular}
%   \end{threeparttable}
% \end{table}

% \begin{comment}
%         Этот текст всегда скрыт.
% \end{comment}
