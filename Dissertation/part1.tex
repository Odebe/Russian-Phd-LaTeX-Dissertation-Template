\chapter{Описание прикладной области}\label{ch:ch1}
\section{Интернет-хостинг}\label{sec:ch1/sec1}
Интернет-хостинг - это служба, которая предоставляет интернет-серверы, позволяя организациям и частным лицам размещать контент в Интернет. Существуют различные уровни обслуживания и различные виды предлагаемых услуг.

Распространенным видом хостинга является веб-хостинг. Большинство хостинг-провайдеров предлагают комбинацию услуг, например, хостинг электронной почты.

Чаще всего интернет-хостинг предоставляет сервер с хорошей пропуской способностью интернет соединения, на котором клиенты могут запускать любые приложения.

\subsection{Dedicated Server. Выделенный сервер.}\label{sec:ds_hosting}
Выделенный хостинг - это вариант хостинга в Интернете, в котором физический сервер (или серверы) выделяются для одного бизнес-клиента. Заказчик имеет полный контроль над машиной, поэтому он может оптимизировать ее для своих уникальных требований, включая производительность и безопасность. Хостинг-провайдер предоставляет физический сервер и среду, сопутствующие услуги и техническую поддержку.

\subsection{VPS. Виртуальный выделенный сервер}\label{sec:vps_hosting}
VPS - это сокращение от Virtual Private Server (Виртуальный выделенный сервер). VPS хостинг - одна из самых популярных услуг хостинга, которую можно выбрать для размещения сайта и приложений. Для работы используется технология виртуализации, позволяющая предоставить выделенные (частные) ресурсы на сервере с несколькими пользователями.

Это более безопасное и стабильное решение, чем виртуальный хостинг, где клиент не получает выделенного серверного пространства. Тем не менее, VPS меньше и дешевле, чем аренда всего сервера.

Хостинг VPS обычно выбирают владельцы веб-сайтов, у которых трафик среднего уровня превышает лимиты планов общего хостинга, но при этом ему не нужны ресурсы выделенного сервера, или энтузиасты, которые ипользуются выделенные сервера для своих приложений.

\subsection{Колокация}\label{sec:colo_hosting}
Колокация - услуга, состоящая в том, что клиенту предоставляется оборудование на своей территории (датацентре). В случаях, если оборудование тоже арендуется клиентом, это называется "арендой выделенного сервера".

Такое размещение позволяет сэкономить на организации связи. Чаще всего колокацию используются для веб-сайтов и другим сетевым службам с большим объемом трафика, а также оборудование, к которму необходимое  доступ из многих точкек, например, VPN-концентраторы, шлюзы IP-телефонии.

\subsection{Облачный хостинг}\label{sec:cloud_hosting}
Облачный хостинг - это тип веб-хостинга, который использует несколько разных серверов для балансировки нагрузки и максимизации времени безотказной работы. Вместо использования одного сервера веб-сайт может подключиться к «кластеру», который использует ресурсы из централизованного пула. Это означает, что даже если один сервер выходит из строя, другой начинает работать, что повышает отказоуствойчивость системы.

Визуализируйте облако можно представить сеть взаимосвязанных компьютеров. Чем больше компьютеров подключено в сети, тем больше ресурсов добавляется в общее облако.

С облачным хостингом клиент получает часть так называемого облачного кластера. В отличие от традиционного веб-хостинга, где клиент получает определенное количество места с одного сервера.

Основные преимущества облачного хостинга заключаются во времени безотказной работы, изолированности ресурсов, простоте масштабирования и выделенных внешних адресах.

\section{Виртуализация}\label{sec:ch1/sec2}
\subsection{Описание}\label{sec:virt_decs}
Виртуализация использует программное обеспечение для создания уровня абстракции над компьютерным оборудованием, которое позволяет разделить аппаратные элементы одного компьютера - процессоры, память, хранилище и т.д. - на несколько виртуальных компьютеров, обычно называемых виртуальными машинами (ВМ). Каждая виртуальная машина работает со своей собственной операционной системой (ОС) и ведет себя как независимый компьютер, даже если она работает только на части фактического компьютерного оборудования.

Из этого следует, что виртуализация позволяет более эффективно использовать физическое компьютерное оборудование и обеспечивает большую отдачу от вложений в оборудование организации.

Сегодня виртуализация является стандартной практикой в корпоративной ИТ-архитектуре. Это также технология, которая управляет экономикой облачных вычислений. Виртуализация позволяет пользователям облачных вычислений приобретать только те вычислительные ресурсы, которые им необходимы, когда им они необходимо, и экономически эффективно масштабировать эти ресурсы по мере роста их рабочих нагрузок.

\subsection{Объекты виртуализации}\label{sec:virt_objs}
\textbf{Операционные системы.} 
Виртуализация операционных систем — это метод размещения множества «гостевых» операционных систем (guest) поверх одной операционной системы-«хозяина» (host). Изолированные образы ОС называют «контейнерами», «виртуальными частными серверами» (Virtual Private Server, VPS) или «виртуальными средами» (Virtual Environment,VE). С точки зрения хоста VPS или VE выглядит как настоящий сервер. Этот тип виртуализации реализован во многих операционных системах: IBM AIX (технология WPARs), HP-UX (HP vPars), Sun Solaris (Container/Zone), FreeBSD(FreeBSD Jail), Linux (Linux-VServer, OpenVZ, Virtuozzo Containers), Windows (Virtuozzo Containers).

\textbf{Приложения.}
Виртуализация приложений преследует своей целью отделить приложения от операционной системы, сделать их мобильными и предоставить возможность для выполнения приложений в различных средах, например, на клиентском рабочем месте. Сравним: в обычной среде приложения инсталлируются непосредственно в операционную систему. Поскольку они используют общие системные ресурсы, то между ними есть вероятность возникновения конфликтов, что может приводить к нестабильности в работе ОС и даже к ее аварийному завершению. Эти негативные проявления полностью исключены, если каждое из приложений работает в защитной среде-оболочке.

\textbf{Оперативная память.}
Виртуализация оперативной памяти является обязательным условием виртуализации любого компьютера, поскольку позволяет динамически распределять память между виртуальными машинами, работающими на одной системе, под управлением одного монитора. Она устанавливает соответствие между виртуальными адресами, которые используют гостевые ОС и физическими адресами памяти машины. Аналогичную роль играет виртуализация периферийных устройств и ввода/вывода.

\textbf{Системы хранения.}
Виртуализация систем хранения — отдельная тема, она известна давно и хорошо. В нашем журнале было опубликовано несколько обзорных статей по этой проблеме.

\textbf{Данные.}
Для виртуализации данных существуют иные названия — Information-as-a-Service и Data-as-a-Service.

\textbf{Рабочие места.}
Известно множество самых разнообразных способов виртуализации, предназначенных для виртуализации рабочих мест, среди них наибольшее распространение получили несколько основных. Во-первых, удаленный доступ по сети (Single Remote Desktop) с использованием технологий PCAnywhere, Windows Remote Desktop и других. Во-вторых, Shared Desktops, модель, при которой мощный сервер с использованием технологий Citrix, Ericom Software или Terminal Services превращается в среду, на которой можно выполнять сотни и больше терминальных сессий. В-третьих, рабочему месту может быть поставлена в соответствие «виртуальная машина» (Virtual Machine Desktop). 

\subsection{Полная эмуляция}\label{sec:full_amu}
Данный вид виртуализации полностью симулирует всё аппаратное обеспечение при сохранении гостевой операционной системы в неизменном виде. При таком типе виртуализации имеется возможность запускать различные аппаратные архитектуры. Нетрудно догадаться, что данный тип эмуляции существенно нагружает вычислительные ресурсы хостовой системы, что делает работу с ней очень медленной, поэтому, данная техника используется в основном для разработки системного программного обеспечения, а также образовательных целей.

Примеры продуктов для создания эмуляторов: Bochs, PearPC, QEMU (без ускорения), Hercules Emulator.

\subsection{Частичная эмуляция (нативная виртуализация)}\label{sec:part_emu}
В этом случае виртуальная машина эмулирует лишь необходимое количество аппаратного обеспечения, чтобы она могла быть запущена изолированно. Естесственно, что подход позволяет запускать гостевые операционные системы, разработанные только для той же архитектуры, что и у хоста. Этот вид виртуализации положительно влияет на быстродействие гостевых систем по сравнению с полной эмуляцией и широко используется в настоящее время. Для увеличения быстродействия также используется следующий подход, применяется специальная «прослойка» между гостевой операционной системой и оборудованием (гипервизор), позволяющая гостевой системе напрямую обращаться к ресурсам аппаратного обеспечения. Гипервизор, называемый также «Монитор виртуальных машин» (Virtual Machine Monitor) - одно из основных понятий в мире виртуализации. Применение гипервизора существенно увеличивает быстродействие платформы, приближая его к быстродействию физической платформы. Самым большим минусом данного вида виртуализации можно отнести зависимость виртуальных машин от архитектуры аппаратной платформы.

Примеры продуктов для нативной виртуализации: VMware Workstation, VMware Server, VMware ESX Server, Virtual Iron, Virtual PC, VirtualBox, Parallels Desktop и другие.

\subsection{Виртуализация адресного пространства}\label{sec:virt_space_emu}
В данном случае, виртуальная машина симулирует несколько экземпляров аппаратного окружения. Притаком подходе имеется возможность совместно использовать ресурсы и изолировать процессы, но не позволяет разделять экземпляры гостевых операционных систем. В данном виде виртуальные машину не создаются полностью, а происходит изоляция каких-либо процессов на уровне операционной системы. В данный момент многие из известных операционных систем используют такой подход.

\subsection{Паравиртуализация}\label{sec:paravirt}
При применении паравиртуализации нет необходимости симулировать аппаратное обеспечение, используется специальный программный интерфейс (API) для взаимодействия с гостевой операционной системой. При данном подходе требуется модифицированное гостевое ядро системы. Так как при построение данного вида виртуализированного окружения необходимо модифицировать код ядра операционной системы, повсеместное внедрение данного типа не получило широкого распространения по понятным причинам.

\subsection{Виртуализация уровня операционной системы}\label{sec:os_level_virt}
Сутью данного вида виртуализации является виртуализация физического сервера на уровне операционной системы в целях создания нескольких защищенных виртуализованных серверов на одном физическом. Гостевая система, в данном случае, разделяет использование одного ядра хостовой операционной системы с другими гостевыми системами. Виртуальная машина представляет собой окружение для приложений, запускаемых изолированно. Данный тип виртуализации применяется при организации систем хостинга, когда в рамках одного экземпляра ядра требуется поддерживать несколько виртуальных серверов клиентов.

Примеры виртуализации уровня ОС: Linux-VServer, Virtuozzo, OpenVZ, Solaris Containers и FreeBSD Jails.

\subsection{Виртуализация уровня приложений}\label{sec:app_level_virt}
Данный вид виртуализации не похож на все остальные: в отличии от других способов само приложение помещается в контейнер с необходимыми элементами для своей работы: файлами реестра, конфигурационными файлами, пользовательскими и системными объектами. В результате получается приложение, не требующее установки на аналогичной платформе. При переносе такого приложения на другую машину и его запуске, виртуальное окружение, созданное для программы, разрешает конфликты между ней и операционной системой, а также другими приложениями. Такой способ виртуализации похож на поведение интерпретаторов различных языков программирования (недаром интерпретатор, Виртуальная Машина Java (JVM), тоже попадает в эту категорию).

Примером такого подхода служат: Thinstall, Altiris, Trigence, Softricity. 

\section{Обзор существующих решений}\label{sec:ch1/sec3}
\subsection{Amazon VPS}\label{sec:amazon}
Amazon Virtual Private Cloud (Amazon VPC) – это логически изолированный раздел облака Amazon Web Services (AWS), в котором можно запускать ресурсы AWS в самостоятельно заданной виртуальной сети. Таким образом можно полностью контролировать среду виртуальной сети, в том числе выбирать собственный диапазон IP‑адресов, создавать подсети, а также настраивать таблицы маршрутизации и сетевые шлюзы. Для обеспечения удобного и безопасного доступа к ресурсам и приложениям в VPC можно использовать как протокол IPv4, так и протокол IPv6.
\newline
Amazon VPS предлагает огромное количество тарифов и конфигураций, в таблице~\ref{tab:awazon_vps} описаны некоторые из них.

\begin{table} [htbp]%
  \centering
  \begin{threeparttable}% выравнивание подписи по границам таблицы
    \caption{Список некоторых тарифов ВМ Amazon VPS}
    \label{tab:awazon_vps}% label всегда желательно идти после caption
    \renewcommand{\arraystretch}{1}%% Увеличение расстояния между рядами, для улучшения восприятия.
    \begin{SingleSpace}
      \begin{tabular}{@{}@{\extracolsep{20pt}}llll@{}}
        % Вертикальные полосы не используются принципиально, как и лишние горизонтальные (допускается по ГОСТ 2.105 пункт 4.4.5) % @{} позволяет прижиматься к краям
        \toprule     %%% верхняя линейка
          Название & Вирт.ЦПУ & Память ГиБ & Цена за час \\
        \midrule %%% тонкий разделитель. Отделяет названия столбцов. Обязателен по ГОСТ 2.105 пункт 4.4.5
          a1.medium & 1 & 2 & 0,0291 USD \\
          a1.large & 2 & 4 & 0,0582 USD \\
          a1.xlarge & 4 & 8 & 0,1164 USD \\
          a1.2xlarge & 8 & 16 & 0,2328 USD \\
          a1.2xlarge & 8 & 16 & 0,2328 USD \\
          a1.4xlarge & 16 & 32 & 0,4656 USD \\
          a1.metal & 16 & 32 & 0,466 USD \\
          t3.nano & 2 & 0,5 & 0,006 USD \\
          t3.micro & 2 & 1 & 0,012 USD \\
          t3.small & 2 & 2 & 0,024 USD \\
          t3.large & 2 & 8 & 0,096 USD \\
        \bottomrule %%% нижняя линейка
      \end{tabular}%
    \end{SingleSpace}
  \end{threeparttable}
\end{table}

\subsection{Hostwinds}\label{sec:hostwinds}
Компания прдоставляет большой спект услуг - веб-хостинг, облачные вычисления, выделенные физические и виртуальные сервера, как с linux, так и с windows. Основана в 2010 году, сервера находятся в Сиэтле, Далласе и Амстердаме. 
\newline
В таблице~\ref{tab:hostwinds_vps} описаны некоторые конфигурации.

\begin{table} [htbp]%
  \centering
  \begin{threeparttable}% выравнивание подписи по границам таблицы
    \caption{Список некоторых тарифов ВМ Hostwinds}
    \label{tab:hostwinds_vps}% label всегда желательно идти после caption
    \renewcommand{\arraystretch}{1}%% Увеличение расстояния между рядами, для улучшения восприятия
    \begin{SingleSpace}
      \begin{tabular}{@{}@{\extracolsep{10pt}}lllll@{}}
        % Вертикальные полосы не используются принципиально, как и лишние горизонтальные (допускается по ГОСТ 2.105 пункт 4.4.5) % @{} позволяет прижиматься к краям
        \toprule     %%% верхняя линейка
          Вирт.ЦПУ & Память ГиБ & Хранилище ГиБ & Трафик ТБ & Цена за месяц \\
        \midrule %%% тонкий разделитель. Отделяет названия столбцов. Обязателен по ГОСТ 2.105 пункт 4.4.5
          1 & 1 & 30 & 1 & 4,49 USD \\
          1 & 2 & 50 & 2 & 8,99 USD \\
          2 & 4 & 75 & 2 & 17,09 USD \\
          2 & 6 & 100 & 2 & 26,09 USD \\
          4 & 8 & 150 & 3 & 36,09 USD \\
        \bottomrule %%% нижняя линейка
      \end{tabular}%
    \end{SingleSpace}
  \end{threeparttable}
\end{table}

\subsection{Ionos}\label{sec:ionos}
IONOS является веб-хостингом и провайдеров облаков для малого и среднего бизнеса. Компания являемся экспертами в области IaaS и предлагает портфель решений для цифрового пространства. 
Компания предоставляет большое количество услуг - регистрация доменов, конструктор сайтов, хостинг сайтов, выделенных и виртуальных серверов, облачных хранилищ.
\newline

В таблице~\ref{tab:ionos_vps} описаны некоторые конфигурации.
\begin{table} [htbp]%
  \centering
  \begin{threeparttable}% выравнивание подписи по границам таблицы
    \caption{Список некоторых тарифов ВМ Ionos}
    \label{tab:ionos_vps}% label всегда желательно идти после caption
    \renewcommand{\arraystretch}{1}%% Увеличение расстояния между рядами, для улучшения восприятия
      \begin{tabular}{@{}@{\extracolsep{10pt}}lllll@{}}
        % Вертикальные полосы не используются принципиально, как и лишние горизонтальные (допускается по ГОСТ 2.105 пункт 4.4.5) % @{} позволяет прижиматься к краям
        \toprule     %%% верхняя линейка
          Название & Вирт.ЦПУ & Память ГиБ & Хранилище ГиБ & Цена за месяц \\
        \midrule %%% тонкий разделитель. Отделяет названия столбцов. Обязателен по ГОСТ 2.105 пункт 4.4.5
          VPS S & 1 & 0,5 & 10 & 2 USD \\
          VPS M & 2 & 2 & 80 & 10 USD \\
          VPS L & 2 & 4 & 120 & 20 USD \\
          VPS XL & 4 & 8 & 160 & 30 USD \\
          VPS XXL & 6 & 12 & 240 & 30 USD \\
        \bottomrule %%% нижняя линейка
      \end{tabular}%
  \end{threeparttable}
\end{table}

\subsection{Alibabacloud}\label{sec:alibabacloud}
Компания Alibaba Cloud была основана в сентябре 2009 года. Она разрабатывает платформы для облачных вычислений и управления данными и предоставляет полный набор услуг в сфере облачных вычислений для поддержки предприятий во всем мире. Является подразделением крупнейшей китайской интернет-компании Alibaba Group. К маю 2018 году у Alibaba Cloud насчитывается несколько партнеров в России.
Услуги предоставляемые компание относятся к областям облачных вычислений, облачных хранилищ, аналитики, искуственного интеллекта, интернета вещей и платформ для разработчиков.
\newline

В таблице~\ref{tab:alibabacloud_vps} описаны некоторые конфигурации.
\begin{table} [htbp]%
  \centering
  \begin{threeparttable}% выравнивание подписи по границам таблицы
    \caption{Список некоторых тарифов ВМ Alibabacloud}
    \label{tab:alibabacloud_vps}% label всегда желательно идти после caption
    \renewcommand{\arraystretch}{1}%% Увеличение расстояния между рядами, для улучшения восприятия
    \begin{SingleSpace}
      \begin{tabular}{@{}@{\extracolsep{10pt}}lllll@{}}
        % Вертикальные полосы не используются принципиально, как и лишние горизонтальные (допускается по ГОСТ 2.105 пункт 4.4.5) % @{} позволяет прижиматься к краям
        \toprule     %%% верхняя линейка
          Вирт.ЦПУ & Память ГиБ & Хранилище ГиБ & Трафик ТБ & Цена за месяц \\
        \midrule %%% тонкий разделитель. Отделяет названия столбцов. Обязателен по ГОСТ 2.105 пункт 4.4.5
          1 & 0,5 & 20 & 1 & 3,3 USD \\
          1 & 1 & 25 & 2 & 4,5 USD \\
          1 & 2 & 50 & 3 & 9,0 USD \\
          2 & 2 & 60 & 3 & 13,5 USD \\
          2 & 4 & 80 & 4 & 18,0 USD \\
          2 & 8 & 80 & 5 & 36,0 USD \\
          4 & 16 & 320 & 6 & 72,0 USD \\
          8 & 32 & 500 & 7 & 144,0 USD \\
        \bottomrule %%% нижняя линейка
      \end{tabular}%
    \end{SingleSpace}
  \end{threeparttable}
\end{table}

\subsection{Hetzner}\label{sec:hetzner}
TODO